% !TEX program = pdflatex
\documentclass[12pt]{article}

% --------- Paquetes básicos ----------
\usepackage[spanish,es-noshorthands]{babel}
\usepackage[utf8]{inputenc}
\usepackage[T1]{fontenc}
\usepackage{lmodern}

% --------- Formato ----------
\usepackage[a4paper,margin=2.5cm]{geometry}
\usepackage{setspace}
\onehalfspacing
\usepackage{parskip}

% --------- Matemáticas ----------
\usepackage{amsmath,amssymb}

% --------- Figuras y tablas ----------
\usepackage{graphicx}
\usepackage{float}
\usepackage{booktabs}
\usepackage{longtable}
\usepackage{siunitx}

% --------- Código ----------
\usepackage{listings}
\usepackage{xcolor}
\lstdefinestyle{juliastyle}{
  language={[Julia]TeX},
  basicstyle=\ttfamily\small,
  keywordstyle=\color{blue},
  commentstyle=\color{gray},
  stringstyle=\color{teal},
  numbers=left,
  numberstyle=\tiny\color{gray},
  stepnumber=1,
  numbersep=8pt,
  showstringspaces=false,
  breaklines=true,
  frame=single,
  tabsize=2
}
\lstdefinelanguage{Julia}{
  morekeywords={
    abstract,break,case,catch,const,continue,do,else,elseif,end,export,false,finally,for,function,
    global,if,import,in,let,local,macro,module,mutable,primitive,quote,return,struct,true,try,using,
    while,where
  },
  sensitive=true,
  morecomment=[l]\#,
  morestring=[b]",
}

\lstdefinestyle{juliastyle}{
  language=Julia,
  basicstyle=\ttfamily\small,
  keywordstyle=\color{blue},
  commentstyle=\color{gray},
  stringstyle=\color{teal},
  numbers=left,
  numberstyle=\tiny\color{gray},
  stepnumber=1,
  numbersep=8pt,
  showstringspaces=false,
  breaklines=true,
  frame=single,
  tabsize=2
}

\lstset{style=juliastyle}
% --------- Hipervínculos ----------
\usepackage[hidelinks]{hyperref}

% --------- Título ----------
\title{\textbf{Comparación de Órdenes de Crecimiento y Simulación de Costos}}
\author{Javier Jhairt López Rojas}
\date{\today}

\begin{document}
\maketitle

% =========================================================
\section{Introducción}
% Qué es un orden de crecimiento, por qué importa, qué se hará en el reporte.
La siguiente solución muestra una explicación exhaustiva de los comportamientos asintoticos de diferentes casos de comportamientos algoritmicos. El objetivo principal es  crear un entendimiento comparativo de diversos escenarios hipoteticos de comportamiento computacional, de tal forma que se pueda implementar en escenarios reales.
% =========================================================
\section{Metodología}
\subsection{Supuestos}



Los supuesttos realizados para el desarrllo de esta practica son los sigueintes: j

\begin{itemize}
    \item Entrada de datos limitada: En un mundo real, algunos algortimos con complejidad demasialo elevada una entrada de datos igual de masiva provoca daños en hardware, por lo que se decidió acotar dichos valores.
    \item Modelo de computo idealizado: Se asume que el hardware funciona de manera ideal, sin interrupciones, errores o fallos.
\end{itemize}

\subsection{Herramientas y entorno}
% Ej: Julia, Jupyter/Quarto, Plots, etc.
Los experimentos se realizaron en Julia (notebook Jupyter/Quarto). Se utilizaron funciones matemáticas y gráficas para comparar visualmente los crecimientos y generar una tabla de tiempos simulados.

% =========================================================
\section{Figuras y comparación de los órdenes de crecimiento}
% Aquí van subsecciones por comparación, con figura + texto interpretativo.

\subsection{$O(1)$ vs $O(\log n)$}
% Inserta figura exportada como PNG/PDF desde Julia
\begin{figure}[H]
    \centering
    \caption{Comparación entre $O(1)$ y $O(\log n)$ en un rango seleccionado de $n$.}
    \includegraphics[width=\textwidth]{images/1-nlogn.png}
    \label{fig:o1_olog}
\end{figure}

% Discusión del rango y comportamiento
Discusión: El rango usado para esta comparación fue de [0-300]. Podemos notar que a medida que n crece el logaritmo cada ves mas converge a un comportamiento mas estable dado que la función logartimo tiene la propiedad de llevar numeros muy grandes a cantidades mas suaves.

\subsection{$O(n)$ vs $O(n\log n)$}
\begin{figure}[H]
    \centering
    \caption{Comparación entre $O(n)$ y $O(n\log n)$.}
    \includegraphics[width=\textwidth]{images/n-nlogn.png}
    \label{fig:on_onlog}
\end{figure}

Discusión: el rango usado fue de [0-300]. Se peude apreciar que a medida que n crece, el comportamiento asintótico de $n \log n$ es mucho mas inmenso que un comportamiento lineal, ésto porque a diferencia de la funcion $\log$, tiene un factor n que lo potencia a tener un mayor creciemiento.

\subsection{$O(n^2)$ vs $O(n^3)$}
\begin{figure}[H]
    \centering
    \caption{Comparación entre $O(n^2)$ y $O(n^3)$.}
    \includegraphics[width=\textwidth]{images/n2-n3.png}
    \label{fig:on2_on3}
\end{figure}

Discusión: Es interesante opbswervar que la función $n^2$ puede verse “menos curvo” si $n^3$ domina el eje $y$, ésto debido a que $n^3$ crece mucho mas rapido que $n^2$.

\subsection{$O(a^n)$ vs $O(n!)$}
\begin{figure}[H]
    \centering
    \caption{Comparación entre $O(a^n)$ y $O(n!)$ (por ejemplo $a=2$).}
    \includegraphics[width=\textwidth]{images/an-n!.png}
    \label{fig:oan_onfact}
\end{figure}

Discusión: Debido a que ambos crecen de manera muy abrupta, se decidio usar un rango de [0-20] para poder apreciar el comportamiento de ambas funciones.

\subsection{$O(n!)$ vs $O(n^n)$}
\begin{figure}[H]
    \centering
    \caption{Comparación entre $O(n!)$ y $O(n^n)$.}
    \includegraphics[width=\textwidth]{images/n!-nn.png}
    \label{fig:onfact_onn}
\end{figure}

Discusión: La funcion $n^n$ crece demasiado rapido incluso con valores pequeños.

% =========================================================
\section{Análisis y simulación de costo en formato de tabla}
% Aquí debes poner la tabla final con tiempos simulados.

\subsection{Tabla de tiempos simulados}
Se calcula el número de operaciones para distintos tamaños de entrada $n$ y se convierte a tiempo suponiendo \SI{1}{\nano\second} por operación.

% Opción A: tabla corta (si cabe)
\begin{longtable}{@{}llS[table-format=3.2e2]l@{}}
    \caption{Simulación de costos computacionales (1 operación = 1 ns).}\label{tab:sim_costos} \\
    \toprule
    Orden          & $n$   & {Operaciones (aprox.)} & Tiempo (aprox.)                          \\ \midrule
    \endfirsthead
    \toprule
    Orden          & $n$   & {Operaciones (aprox.)} & Tiempo (aprox.)                          \\ \midrule
    \endhead
    \bottomrule
    \endfoot
    $O(1)$         & 10    & 1                      & \SI{1}{\nano\second}                     \\
                   & 100   & 1                      & \SI{1}{\nano\second}                     \\
                   & 1000  & 1                      & \SI{1}{\nano\second}                     \\
                   & 10000 & 1                      & \SI{1}{\nano\second}                     \\ \midrule
    $O(\log n)$    & 10    & 3.32                   & \SI{3.32}{\nano\second}                  \\
                   & 100   & 6.64                   & \SI{6.64}{\nano\second}                  \\
                   & 1000  & 9.97                   & \SI{9.97}{\nano\second}                  \\
                   & 10000 & 13.29                  & \SI{13.29}{\nano\second}                 \\ \midrule
    $O(\sqrt{n})$  & 10    & 3.16                   & \SI{3.16}{\nano\second}                  \\
                   & 100   & 10.0                   & \SI{10.0}{\nano\second}                  \\
                   & 1000  & 31.62                  & \SI{31.62}{\nano\second}                 \\
                   & 10000 & 100.0                  & \SI{100.0}{\nano\second}                 \\ \midrule
    $O(n^2)$       & 10    & 100                    & \SI{100}{\nano\second}                   \\
                   & 100   & 10000                  & \SI{10}{\micro\second}                   \\
                   & 1000  & 1000000                & \SI{1}{\milli\second}                    \\
                   & 10000 & 1.00e8                 & \SI{0.1}{\second}                        \\ \midrule
    $O(2^n)$       & 10    & 1024                   & \SI{1.02}{\micro\second}                 \\
                   & 20    & 1.05e6                 & \SI{1.05}{\milli\second}                 \\
                   & 30    & 1.07e9                 & \SI{1.07}{\second}                       \\
                   & 50    & 1.13e15                & $\approx$ \SI{13.03}{días}               \\ \midrule
    $O(n!)$        & 5     & 120                    & \SI{120}{\nano\second}                   \\
                   & 10    & 3.63e6                 & \SI{3.63}{\milli\second}                 \\
                   & 20    & 2.43e18                & $\approx$ \SI{77.15}{años}               \\
                   & 50    & {\text{MUY GRANDE}}    & {\text{MUY GRANDE}}                      \\ \midrule
    $O(n^n)$       & 5     & 3125                   & \SI{3.13}{\micro\second}                 \\
                   & 10    & 1.00e10                & \SI{10}{\second}                         \\
                   & 20    & 1.05e26                & $\approx$ \SI{3.33e9}{años}              \\
                   & 50    & {\text{MUY GRANDE}}    & {\text{MUY GRANDE}}                      \\ \midrule
    $O(n^{(n^n)})$ & 2     & 16                     & \SI{16}{\nano\second}                    \\
                   & 3     & 7.63e12                & $\approx$ \SI{2.12}{h}                   \\
                   & 4     & {\text{MUY GRANDE}}    & {\text{MUY GRANDE}}                      \\
                   & 5     & {\text{MUY GRANDE}}    & {\text{MUY GRANDE}}                      \\
\end{longtable}

% =========================================================
\section{Conclusiones}

Dados los resultados de las pruebas desarrolladas en esta practica, podemos observar distintos escenarios que nos peuden ayudar a detectar cuando un algoritmo es mas eficiente que otro con condiciones dadas, de esta forma podemos concluir que un factor importante para observar cambios en elk comportamiento de un algoritmo es la entrada de datos, ya que en anlgunos intervalos finos se peuden observar comportamientos muy diferentes con respecto a su mismo comportamiento con numeros mas largos.

\end{document}
