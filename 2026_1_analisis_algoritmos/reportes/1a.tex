% !TEX program = pdflatex
\documentclass[12pt]{article}

% --------- Paquetes básicos ----------
\usepackage[spanish,es-noshorthands]{babel}
\usepackage[utf8]{inputenc}
\usepackage[T1]{fontenc}
\usepackage{lmodern}

% --------- Formato ----------
\usepackage[a4paper,margin=2.5cm]{geometry}
\usepackage{setspace}
\onehalfspacing
\usepackage{parskip}

% --------- Matemáticas ----------
\usepackage{amsmath,amssymb}

% --------- Figuras y tablas ----------
\usepackage{graphicx}
\usepackage{float}
\usepackage{booktabs}
\usepackage{longtable}
\usepackage{siunitx}

% --------- Código ----------
\usepackage{listings}
\usepackage{xcolor}
\lstdefinestyle{juliastyle}{
  language={[Julia]TeX},
  basicstyle=\ttfamily\small,
  keywordstyle=\color{blue},
  commentstyle=\color{gray},
  stringstyle=\color{teal},
  numbers=left,
  numberstyle=\tiny\color{gray},
  stepnumber=1,
  numbersep=8pt,
  showstringspaces=false,
  breaklines=true,
  frame=single,
  tabsize=2
}
\lstdefinelanguage{Julia}{
  morekeywords={
    abstract,break,case,catch,const,continue,do,else,elseif,end,export,false,finally,for,function,
    global,if,import,in,let,local,macro,module,mutable,primitive,quote,return,struct,true,try,using,
    while,where
  },
  sensitive=true,
  morecomment=[l]\#,
  morestring=[b]",
}

\lstdefinestyle{juliastyle}{
  language=Julia,
  basicstyle=\ttfamily\small,
  keywordstyle=\color{blue},
  commentstyle=\color{gray},
  stringstyle=\color{teal},
  numbers=left,
  numberstyle=\tiny\color{gray},
  stepnumber=1,
  numbersep=8pt,
  showstringspaces=false,
  breaklines=true,
  frame=single,
  tabsize=2
}

\lstset{style=juliastyle}
% --------- Hipervínculos ----------
\usepackage[hidelinks]{hyperref}

% --------- Título ----------
\title{\textbf{Comparación de Órdenes de Crecimiento y Simulación de Costos}}
\author{Tu Nombre Aquí}
\date{\today}

\begin{document}
\maketitle

% =========================================================
\section{Introducción}
% Qué es un orden de crecimiento, por qué importa, qué se hará en el reporte.


% =========================================================
\section{Metodología}
\subsection{Supuestos}
% Ej: 1 operación = 1 ns, costos simulados, no benchmarking real.

\subsection{Herramientas y entorno}
% Ej: Julia, Jupyter/Quarto, Plots, etc.
Los experimentos se realizaron en Julia (notebook Jupyter/Quarto). Se utilizaron funciones matemáticas y gráficas para comparar visualmente los crecimientos y generar una tabla de tiempos simulados.

% =========================================================
\section{Figuras y comparación de los órdenes de crecimiento}
% Aquí van subsecciones por comparación, con figura + texto interpretativo.

\subsection{$O(1)$ vs $O(\log n)$}
% Inserta figura exportada como PNG/PDF desde Julia
\begin{figure}[H]
    \centering
    \caption{Comparación entre $O(1)$ y $O(\log n)$ en un rango seleccionado de $n$.}
    \label{fig:o1_olog}
\end{figure}

% Discusión del rango y comportamiento
Discusión: justificar el rango usado y explicar por qué $\log n$ crece lentamente.

\subsection{$O(n)$ vs $O(n\log n)$}
\begin{figure}[H]
    \centering
    \caption{Comparación entre $O(n)$ y $O(n\log n)$.}
    \label{fig:on_onlog}
\end{figure}

Discusión: diferencias observables y cuándo se vuelven relevantes.

\subsection{$O(n^2)$ vs $O(n^3)$}
\begin{figure}[H]
    \centering
    \caption{Comparación entre $O(n^2)$ y $O(n^3)$.}
    \label{fig:on2_on3}
\end{figure}

Discusión: efecto de escala; por qué $n^2$ puede verse “menos curvo” si $n^3$ domina el eje $y$.

\subsection{$O(a^n)$ vs $O(n!)$}
\begin{figure}[H]
    \centering
    \caption{Comparación entre $O(a^n)$ y $O(n!)$ (por ejemplo $a=2$).}
    \label{fig:oan_onfact}
\end{figure}

Discusión: crecimiento explosivo; rangos pequeños por límites numéricos y físicos.

\subsection{$O(n!)$ vs $O(n^n)$}
\begin{figure}[H]
    \centering
    \caption{Comparación entre $O(n!)$ y $O(n^n)$.}
    \label{fig:onfact_onn}
\end{figure}

Discusión: comparar magnitudes; interpretación física (p.ej. operaciones vs edad del universo).

% =========================================================
\section{Análisis y simulación de costo en formato de tabla}
% Aquí debes poner la tabla final con tiempos simulados.

\subsection{Tabla de tiempos simulados}
Se calcula el número de operaciones para distintos tamaños de entrada $n$ y se convierte a tiempo suponiendo \SI{1}{\nano\second} por operación.

% Opción A: tabla corta (si cabe)
\begin{table}[H]
    \centering
    \caption{Tiempos simulados (1 operación = 1 ns).}
    \label{tab:sim_costos}
    \begin{tabular}{@{}llrr@{}}
        \toprule
        Orden       & $n$ & Operaciones (aprox.) & Tiempo (aprox.)      \\ \midrule
        $O(1)$      & 10  & 1                    & \SI{1}{\nano\second} \\
        $O(\log n)$ & 10  & $\log_2(10)$         & $\log_2(10)$ ns      \\
        % Agrega filas según tu salida
        \bottomrule
    \end{tabular}
\end{table}

% Opción B: longtable si es grande (descomenta si lo necesitas)
% \begin{longtable}{@{}llp{5cm}p{5cm}@{}}
% \caption{Tiempos simulados (1 operación = 1 ns).}\label{tab:sim_costos_long}\\
% \toprule
% Orden & $n$ & Operaciones (aprox.) & Tiempo (aprox.) \\ \midrule
% \endfirsthead
% \toprule
% Orden & $n$ & Operaciones (aprox.) & Tiempo (aprox.) \\ \midrule
% \endhead
% \bottomrule
% \endfoot
% % Filas...
% \end{longtable}

% =========================================================
\section{Conclusiones}
% Debe abordar comparaciones y simulación
==============================================
% (Opcional) Referencias
% \section{Referencias}
% \begin{thebibliography}{9}
% \bibitem{cormen} Cormen et al., \textit{Introduction to Algorithms}.
% \end{thebibliography}

\end{document}
