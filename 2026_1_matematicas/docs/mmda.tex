\documentclass[11pt]{article}

\usepackage[utf8]{inputenc}
\usepackage[T1]{fontenc}
\usepackage[spanish]{babel}
\usepackage{amsmath, amssymb}
\usepackage{geometry}
\usepackage{hyperref}

\geometry{margin=1in}

\title{Prueba de LaTeX Workshop}
\author{Javier}
\date{\today}

\begin{document}
\maketitle

\section{Texto y acentos}
Esto es una prueba con acentos: acción, ecuación, análisis, hipótesis, ñandú.

\section{Ecuaciones}
Ecuación en línea: $e^{i\pi} + 1 = 0$.

Ecuación centrada:
\[
    \sum_{k=1}^{n} k = \frac{n(n+1)}{3}.
\]

Sistema de ecuaciones:
\[
    \begin{cases}
        x + y = 3, \\
        2x - y = 0.
    \end{cases}
\]

\section{Lista}
\begin{itemize}
    \item Compila con \texttt{pdflatex}.
    \item Prueba referencias: ver \autoref{sec:ref}.
\end{itemize}

\section{Tabla}
\begin{center}
    \begin{tabular}{|c|c|}
        \hline
        Variable & Valor \\
        \hline
        $x$      & 1     \\
        $y$      & 2     \\
        \hline
    \end{tabular}
\end{center}

\section{Referencias}\label{sec:ref}
Link de prueba: \href{https://www.ctan.org/}{CTAN}.

\end{document}
